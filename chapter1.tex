\chapter{Introduction}
%Deep thoughts go here.

%% The ability to perceive and react to sounds in the environment is a key adaptation of many species.
% Many levels of sound-driven behavior
From the startled jump that helps an animal avoid a predator, to the complex call-and-response of vocal communication, the ability to associate sounds with rewarding actions is a key adaptation of many species. 
% In some cases, the neural circuits underlying the behavior are well-studied.
% TODO: References regarding brainstem startle response.
% Brainstem circuits: Davis et al. 1982 
% High-level cortical circuits for perception and production of speech:
% Pulvermuller and Fadiga 2010
For some simple sound-action associations, such as the acoustic startle response, the neural pathways supporting the association are well-studied. 
% In other cases, the pathways involved in choosing appropriate actions in response to sounds are not clear.
However, for behaviors that require learning associations between specific sounds and the action that should be taken to get a reward, the neural pathways are not well understood. 

% How to find circuits that are required for a behavior?
One approach for identifying neural pathways that implement certain behaviors is to begin at the 'chunk-of-brain' level, inactivating or removing certain brain areas to determine which regions are required for task performance.
% Cortex not required for flexible sound-frequency discrimination
In the case of a sound-frequency discrimination task, in which an animal hears a sound and then must make a choice about whether that sound is a ``high'' frequency or a ``low'' frequency, lesion studies have indicated that auditory cortex is not required, pushing the search for the necessary circuitry into subcortical areas \citep{Gimenez2015}.

\section{Cortex is not always required for flexible decision-making about sounds}

Consider the following behavioral task: An animal is required to learn an arbitrary decision boundary between ``high'' and ``low'' sound frequencies, and then correctly categorize sounds as above or below this decision boundary to receive reward.
Then, the decision boundary is varied by the experimenters during the course of a single behavioral session, requiring rapid flexibility on the part of the animal.
Then, once the animal has learned the task, the entire auditory cortex is surgically removed, and the animal is tested to see if it is still able to perform the task.
Taking this approach, \citet{Gimenez2015} found that animals were still able to perform this task to a high degree of accuracy after nearly-complete lesion of auditory cortex.
However, lesions of auditory thalamus completely impaired the ability of animals to perform the task. 
These results suggest that a subcortical output pathway of the auditory thalamus is sufficient to mediate this task.

The auditory thalamus, while often viewed as a stop along the way for information about sounds traveling from the periphery to the auditory cortex, has several subcortical outputs as well.
% TODO: Citations for 3 main subcortical outputs of auditory thalamus.
The main subcortical projection targets for auditory thalamus neurons are the thalamic reticular nucleus, the amygdala, and the striatum \citep{Pinault2004, Bartlett2013}.
Of these targets, the amygdala and the striatum are the most likely candidates for forming associations between sounds and actions.
% TODO: Citations for thalamic reticular nucleus
The thalamic reticular nucleus is composed primarily of inhibitory neurons that provide a recurrent inhibitory circuit back onto neurons in the auditory thalamus. 
% DONE: Citations for amygdala fear learning to auditory stimuli.
The amygdala has been extensively studied in the context of fear conditioning, and much is known about how amygdalar circuits change during learning to associate auditory stimuli with noxious stimuli and promote fear responses \citep{Romanski1992, Rogan1997, Doron1999, Ledoux2000}. 
While these neurons have also been shown to encode both positive and negative valence of sensory cues \citep{Tye2008}, they seem to play only a minimal role in forming associations between stimuli and actions that will bring reward \citep{Baxter2002}.

% TODO: Citations for striatum in context of Parkinson's
The striatum, regions of which have long been studied in the context of motor disorders such as Parkinson's disease, has also been shown to play a key role in reward-related learning \citep{Wickens2003, Balleine2007}.
%% Regions of the striatum, in contrast, have been shown to play a key role in reward-related learning 
% TODO: Dopaminergic innervation citations. 
In addition to receiving information about sound from the auditory thalamus and the auditory cortex, it is also the target of extensive dopaminergic innervation from the substantia nigra pars compacta. 
% TODO: Basal ganglia citations here.
Additionally, the striatum is a primary input to the basal ganglia, which are involved in the control of voluntary movements.
The striatum is therefore ideally situated to act as a key hub in associating sounds with actions that bring rewards.

\section{The striatum and decision making}
% The striatum, historical association with reward and motor function
% TODO: Read a review of striatum/parkinsons stuff and cite. 
Recent studies focusing on the pathway between auditory cortex and the striatum have pointed to a role for this pathway in associating sounds with rewarded actions.
% Petr's paper
First, manipulation of the corticostriatal pathway produces predictable biases in a frequency-discrimination task \citep{Znamenskiy2013}.
% Q's paper
Second, the synaptic efficacy between auditory cortical neurons and their MSN targets changes as animals learn a frequency-discrimination task in a way that could support the learned association \citep{Xiong2015}.
%%%%%%%% CHAPTER 2 PLUG HERE %%%%%%%%%%%%%
Additionally, results of experiments in \ch{\Musc} establish a critical role for posterior striatal circuits in the performance of a sound-frequency discrimination task. 
% TODO: Overview and Zador model
Taken together, these studies point towards a potential model for formation and maintenance of sound-action associations, in which learning strengthens connections between auditory cortex neurons that represent the relevant stimulus and striatal neurons that can promote the rewarded choice.
%
However, this model is inconsistent with the fact that animals can readily perform behaviors based on learned sound-action associations after nearly complete lesions of auditory cortex.
%
This evidence suggests a role for the thalamostriatal pathway as well.

\section{Parallel cortical and subcortical pathways}

% DONE: Little is known? Or was known?
Few studies have investigated the relationship between the parallel corticostriatal and thalamostriatal pathways, and so the relative contribution of these two pathways to sound-action association is still not well established.
% Role of parallel cortical and subcortical pathways in other systems (like fear).
One hypothesis is that the thalamostriatal pathway sends redundant information about sound, facilitating discrimination of simple, tonal stimuli even after extensive cortical lesion.
% Instead we think they send complementary info
% TODO: Do I want to put this more specific hypothesis in the main intro?
We hypothesized instead that information about sounds sent by auditory thalamostriatal and corticostriatal neurons is complementary, with specific sound features differentially encoded by these two pathways.
%%%%%%%% CHAPTER 3 PLUG HERE %%%%%%%%%%%%%
In \ch{\Thstr} we perform pathway-specific electrophysiology to investigate and compare the sound features encoded by these parallel pathways.
%
We find that the two pathways represent some features of sounds, like sound frequency, in a similar way.
%
For other sound features, such as the rate of temporal modulations in amplitude, the two pathways appear to convey complementary stimulus representations to the striatum.
%
These findings suggest that the two pathways may be differentially recruited depending on the relevant features in a task. 

\section{Are the differences in information sent by the two pathways behaviorally relevant?}
% DONE: When would you need the different kind of features? 
% Pull stuff from the review here, etc.
Lesions of auditory cortex have been found to have different effects on performance of behaviors relying on sound-action associations depending on the specific auditory stimulus feature to which the action must be associated.
%
For instance, while rats with bilateral lesions of auditory cortex were able to perform discriminations of sound frequency \citep{Gimenez2015}, a different study found that lesions of AC in gerbils trained to discriminate between fast and slow modulations in sound amplitude strongly impaired task performance \citep{Deutscher2006}.
%
This suggests that the representation of some stimulus features, such as AM rate, that are provided by the cortex may be necessary for successful discrimination of that feature.
%
In contrast, discrimination of sound frequency may be possible with subcortical representations of this feature alone.
%
We hypothesized that inactivation of auditory cortex would have a stronger effect on discrimination of AM rate than on discrimination of sound frequency.
%
To test this hypothesis, we trained animals to perform discriminations of both sound frequency and AM rate within the same behavioral session, allowing us to test the effect of the same manipulation of AC on both discriminations.
%%%%%%%% CHAPTER 4 PLUG HERE %%%%%%%%%%%%%
% DONE: Finish this task
In \ch{\Amod} I present this task and show that animals can be trained to perform the two discriminations during the same behavioral session with a high degree of accuracy.
%

%%%%%%%% CHAPTER 5 PLUG HERE %%%%%%%%%%%%%
% DONE: This sounds like the abstract so far. 
Lastly, in \ch{\Rev} we expand to discuss the role of sensory cortex in behavioral flexibility more broadly.
%
We propose a taxonomy of flexible behaviors, and review the literature that suggests that some types of task/stimulus combinations are dependent on sensory cortex, while others can be mediated by subcortical pathways.
%
% BOOM!
Overall, \ch{\Thstr} provides a framework for studying the convergence of cortical and thalamic information onto the striatum, \ch{\Amod} provides a behavioral task for comparing the relative dependence of discriminations of different stimulus features on information from a specific pathway, and \ch{\Rev} provides a starting point for investigating the relative contributions of cortical and subcortical pathways to flexible behaviors across different sensory modalities. 

%% \section{What does it mean to understand how the brain accomplishes a task?}
%% There are different levels of understanding of a system.
%% We want to understand: (1) What regions of the brain are critical for performance of sound-guided decision making tasks? (2) What information about sound reaches this region? (3) How does the region choose appropriate actions in response to a sound? (4) What pre-processing of sound information is necessary for the region to form associations between sound features and behaviors that will bring reward? 
