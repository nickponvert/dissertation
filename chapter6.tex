\chapter{Conclusion}

%% Main goals
This dissertation revolves around the following questions: 
\begin{itemize}
\item What are the pathways and brain regions necessary to perform some types of sound-driven behavior? 
\item How is auditory information represented along these pathways, and what transformations are done on this information?
\item What is the purpose of this transformation? Are transformed representations of stimuli necessary for task performance? 
\item What is the role of sensory cortex in flexible behavior more generally? For what types of behaviors are the computations performed by cortical circuits really required?
\end{itemize}

It does not fully address any of these questions, but makes progress on addressing them.

\section{Pathways and regions required for sound-driven behavior}
% TODO: Finish this intro. 
We began by attempting to...
%
We chose to address a narrow range of sound-driven behaviors, in which animals categorize sounds that are presented to them according to some relevant stimulus feature.
%
A study published prior to the start of this dissertation work established that animals were able to perform a previously-learned sound frequency-categorization task even after near-complete lesion of auditory cortex,but not after lesion of auditory thalamus \citep{Gimenez2015}.
%
This suggested that a pathway involving one of the subcortical outputs of the auditory thalamus was involved in this task.
%
We investigated the role of the auditory striatum in the context of this task, and found that impairment of this area severely impaired task performance. 
%
% TODO: Something about how this is consistent with Zador stuff, but the discrepancy between the involvement (but unlikely necessity) of AC in this type of task led us to investigate how the corticostriatal and thalamostriatal compared. 
...

\section{Representation of sound features in auditory inputs to the posterior striatum}
% TODO: We did this why? 
We investigated the representation of two features of sound in the parallel thalamostriatal and corticostriatal pathways.
%
We found that the two pathways encoded sound frequency with similar fidelity, but that the representation of temporal modulations in amplitude differed between the pathways. 
%
We hypothesized that the transformation from a temporal code for amplitude modulation to a rate-coded representation, where cortical neurons were much more likely to fire more spikes in response to a preferred rate, would better allow a downstream area to discriminate AM rate by setting a simple spike-rate threshold. 
%
To test this hypothesis, we designed a behavioral task to evaluate the necessity of auditory cortex on discrimination of both sound frequency and AM rate. 
% TODO: Caveats: Cortex might be more modulated by behavioral parameters, context. 

%TODO: Not totally sure about the wording here. 
\section{What types of stimulus features require cortical processing for efficient discrimination?}

We designed a behavioral task that allows measurement of both sound frequency discrimination performance and AM rate discrimination performance within the same behavioral session.
%
This behavioral paradigm, while challenging for animals to learn, allows the experimenter to compare the effect of one manipulation of cortical activity on both discrimination tasks. 
%
In general, this paradigm is designed to allow experimental comparison of the role of a brain region in two discrimination tasks. 
%
This permits comparison of the relative change in task performance between the two discriminations. 
%
This is important because, even if a task could be performed by subcortical pathways in the absence of cortical input, it is unlikely that cortical inactivation will have no effect on performance of a specific task.
%
Therefore, it is important to compare the relative effect of cortical inactivation on discrimination of more than one stimulus dimension. 

% TODO: Am I arguing that any of the experiments we review in the next section should be repeated with this paradigm to actually learn something?? I don't want to argue that.

% Important to compare two different stimuli for some reason? 
% What does this task give us that other tasks do not? 
% Task 
% Cool stuff you can do with it
% Inactivations (specific here)
% Inactivations (general)

% TODO: Link to behavioral flexibility paper. 
% While we have been interested in the role of cortex as a stimulus feature extractor for simple discriminations, we were also interested in the role of sensory cortex in more complex tasks requiring behavioral flexibility. 
% These experiments did require animals to switch back and forth between the type of stimulus they were discriminating, but otherwise did not require any flexible updating of stimulus-action associations. 
% While we have been focusing heavily on the potential role of auditory cortex in extracting features of auditory stimuli and representing them in a way that would be useful for downstream circuits to perform discrimination on, we also wanted to expand our focus and consider what the role of sensory cortex could be in not only discrimination, but in flexible associations. 
Cortical processing of sensory stimuli to extract features and represent them in a way that is useful for downstream areas...

In my dissertation work I was interested in the relative roles of corticostriatal and thalamostriatal pathways. 
%
I identified some differences in the way that stimuli are encoded in these two pathways, and developed a behavioral task to determine whether these differences in coding are relevant during behavior.
%
However, there are more complicated things that cortex may be doing, especially in tasks that require flexible association of sounds and actions. 
%
We turned to the available literature on the role of sensory cortical regions during behavior to try and address the question: what aspects of flexible, sensory-driven behaviors require sensory cortex? 
%
What aspects of these behaviors could potentially be performed by subcortical pathways alone?

\section{What aspects of stimulus-driven behavior require sensory cortex?}

We first attempted to define a framework for classifying tasks in which flexibility is required. 
%
We then review inactivation studies which suggest that cortex is not always required for flexible adaptations to changes in task demands. 
%
However, sensory cortex does appear to make some types of flexible behaviors, such as reversal of stimulus-action associations, faster. 
%
Also, there is evidence to support the idea that sensory cortex is required to perform discriminations on more complex features of stimuli. 
%
Some flexible behaviors don't appear to be possible after cortical lesions, such as those involving selective attention, pointing to a critical role of cortical circuits in these types of tasks. 


% Several things don't need sensory cortex
% Selective attention requires sensory cortex
% First, we provide a framework for classifying tasks in which flexibility is required. We then present studies in animal models which demonstrate that responses of sensory cortical neurons depend on the expected outcome associated with a stimulus. Last, we discuss inactivation studies which indicate that sensory cortex facilitates behavioral flexibility, but is not always required for adapting to changes in environmental conditions.



% \section{original stuff}
% %% Overview %%
% Cortico-striatal involved, but on a similar task cortex is not necessary once animal has learned. We showed that striatum is required for task performance.
% %
% We then turned to the thalamostriatal pathway, and compared the sound information being sent via that pathway and the corticostriatal pathway. Freq same, AM different.
% %
% This raises the question of whether the thalamostriatal pathway is sufficient for the performance of some kinds of tasks, but not sufficient for others. A review of the literature related to the role of cortex in behavioral flexibility supports this idea.
% %
% We designed and implemented a behavioral task that would allow the effect of a manipulation (such as chemical inactivation of AC) on performance of two discrimination tasks, frequency and AM rate.
% 
% 
% 
% 
% 
% The experiments in \ch{\Musc} indicated a crucial role for the posterior striatum during performance of an auditory discrimination task. 
% %
% Previous literature had suggested a role for the pathway between auditory cortex and the posterior striatum in the context of a similar auditory task, which is consistent with the results of the inactivation experiments.
% %
% However, a model in which the corticostriatal pathway is solely responsible for facilitating behaviors involving discriminations of sound frequency are not consistent with lesion studies, which show that animals have a remarkable ability to perform auditory decision-making tasks, even requiring rapid flexibility, without auditory cortex. 
% %
% We therefore investigated the relationship between the auditory corticostriatal pathway and the parallel thalamostriatal pathway, interested in how these two sources of auditory input to the striatum compare and how they contribute to auditory decision-making tasks under different conditions. 
% 
% We found that the thalamostriatal and corticostriatal pathways convey information about sound frequency to the striatum with similar fidelity, potentially explaining why animals were found to be able to perform frequency discrimination tasks without auditory cortex. 
% %
% We also found that the representation of other sound features differed. 
% %
% Specifically, we found that the corticostriatal and thalamostriatal pathways differed in their representation of AM rate, with corticostriatal neurons more likely to be tuned to particular AM rates, and less likely than thalamostriatal neurons to synchronize their firing to the envelope of the modulation.
% %
% % TODO: Cite Deutscher, whatever else.
% Based on these observations, and studies which suggested that AM discrimination performance suffers after auditory cortical lesion, we hypothesized that this cortical representation of AM rate better allows downstream circuits to perform discrimination of AM rate.
% 
% To test this hypothesis, we designed a task to compare the effect of a cortical manipulation on both AM rate discrimination and frequency discrimination in the same behavior session. 
% %
% Although it was difficult for animals to learn this task, with sufficient training they were able to achieve high levels of performance on both discrimination tasks. 
% %
% Pilot inactivation studies in animals performing this task suggested that AC inactivation impairs both frequency and AM rate discrimination behavior, and allowed us to identify key areas for future improvement of this method that would better enable comparisons of the magnitude of the effect on the performance of each discrimination. 
% %
% Principally, future experiments should better leverage the flexibility of the two-alternative forced-choice paradigm by changing the stimulus sets in order to make each discrimination easier or harder, allowing better matching of pre-inactivation performance between the two modalities.
% %
% 
% 
% 
% I found that inactivation of posterior striatum causes task deficits, and that these task deficits can't easily be explained by motor deficits alone.
% %
% %This is consistent with the other experiments in the paper, which showed that activation of posterior striatal neurons does not directly cause head rotation but does cause behavioral biases if done while the animal is performing a task. 
% %
% %The results of this inactivation are also largely consistent with the results of \citet{Znamenskiy2015}, in that they point to a key role for the auditory striatum in the performance of frequency-discrimination tasks. 
% 
% 
% %Striatal activation, in our hands, resulted in strong contralateral biases regardless of the preferred frequency of the striatal neurons. 
% 
% % TODO: Check supplemental of ref 10 in Lan's paper (I think Petr). 
% Zador lab stimulated AC without frequency-specific and found a contralateral bias.
% 
% Both of these pieces of evidence point towards the MSNs in the pStr acting as drivers of contralateral choice. 
% 
% % Do we see these ``contralateral'' biases only because we are using a left/right task? What about a go/no-go task? 
% 
% How can behavioral and neurophysiological comparisons help reveal the role of a brain area?
% 
% If an area is directly making the decision, then the responses of neurons there should be different when animals make a different choice but the presented sounds are the same. 
% 
% Can test whether the pre-withdrawal activity encodes an animal's choice.
% 
% Both of these have the choice being variable because the stimulus is close to the categorization boundary. So you can also do reversal tasks where the stimulus-action association is clear but changes between contingencies. 
% 
% %%%% Stuff related to Thstr paper
% 
% % What evidence exists for synaptic plasticity at thalamostriatal synapses?
% % Look at thalamoamygdala
% It is not known whether connections between sensory thalamus neurons and neurons in the striatum are capable of the same type of learning-related plasticity observed in corticostriatal synapses \citep{Xiong2015}, although plasticity has been observed between intralaminar thalamus and striatum \citep{Parker2016}.
% %
% One study has shown that simultaneous stimulation of auditory corticostriatal and thalamostriatal neurons leads to long term potentiation at corticostriatal synapses but not at thalamostriatal synapses, suggesting that, if the thalamostriatal pathway is capable of synaptic plasticity, different rules may govern the changes. 
% % TODO: I am here
% Plasticity does occur between neurons in the thalamus and those in the amygdala following fear conditioning, cocaine conditioning \citep{Rich2019}
% 
% % The recording experiments happened in untrained animals. What might happen if animals are trained before? Where would the changes potentially occur? 
% 
% % Do thalamostriatal and corticostriatal inputs converge on the same MSNs?
% 
% % What does it mean for the inputs onto MSNs if the MSNs in posterior striatum may not be the ones actually performing the selection? 
% 
% % TODO: Check out Brice B's paper showing that for simple stimuli the thalamic pathway takes over, but for complex stimuli the reaction times are longer and the cortical pathway seems required. 
% 
% 
% % Might cortex just be serving to reduce complex stimuli to simple rate-tuned representations? 
% 
% %%%%%%%%%%%%% Structure %%%%%%%%%%%%%%%%%
% 
% % Overall in this dissertation we found:
% %% Striatum is for sure a part of the system for performing sound frequency discriminaton discrimination behavior
% %
% %% The thalamostriatal and corticostriatal pathways send complementary information about sounds, but it isn't super clear why.
% %
% %% A review of the role of cortex in flexible behavior suggests that there are some classes of behaviors that can be performed without cortex, but there are some classes of beahviors where cortex is necessary.
% %
% %% We developed a task that can be used to evaluate the side-by-side effects on discrimination of different sound features.
% %
% 
