\chapter{Conclusion}

%% Main goals
%This dissertation revolves around the following questions: 
The work in this dissertation attempts to address the following questions: 
\begin{itemize}
\item What are the pathways and brain regions necessary to perform some types of sound-driven behavior? 
\item How is auditory information represented along these pathways, and what transformations are done on this information?
\item What is the purpose of this transformation? Are transformed representations of stimuli necessary for task performance? 
\item What is the role of sensory cortex in flexible behavior more generally? For what types of behaviors are the computations performed by cortical circuits really required?
\end{itemize}

It does not fully address any of these questions, but makes progress on addressing them.

\section{Pathways and regions required for sound-driven behavior}
We began by investigating the role of the posterior striatum in sound frequency discrimination. 
%
We chose to investigate this region for three reasons. 
%
First, study published prior to the start of this dissertation work established that animals were able to perform a previously-learned sound frequency-categorization task even after near-complete lesion of auditory cortex,but not after lesion of auditory thalamus \citep{Gimenez2015}.
%
This suggested that a pathway involving one of the subcortical outputs of the auditory thalamus was involved in this task.
%
Additionally, a pair of studies suggested that the corticostriatal pathway, under normal conditions, is involved in a similar sound-frequency discrimination task \citep{Znamenskiy201}, and that the synaptic strength between auditory cortex neurons and striatal neurons changes during learning in a way that could support the learned sound-action association \citep{Xiong2015}.
%
This evidence led us to hypothesize that the preservation of the ability to perform frequency discrimination after lesion of auditory cortex was because the parallel thalamostriatal pathway continued to provide frequency information to the posterior striatum.
%%%%%%%%%%

%%%%%%%%%%
We set out to address this hypothesis by first determining whether frequency discrimination is possible after inactivation of posterior striatum. 
%
We found that inactivation of posterior striatum severely impaired frequency discrimination performance, pointing to a key role for this area in the performance of this behavior. 
%
We then attempted to address the role of the parallel thalamostriatal and corticostriatal pathways.

\section{Representation of sound features in auditory inputs to the posterior striatum}
We began by investigating the representation of two features of sound in the parallel thalamostriatal and corticostriatal pathways.
%
We found that the two pathways encoded sound frequency with similar fidelity, but that the representation of temporal modulations in amplitude differed between the pathways. 
%
We hypothesized that the transformation from a temporal code for amplitude modulation to a rate-coded representation, where cortical neurons were much more likely to fire more spikes in response to a preferred rate, would better allow a downstream area to discriminate AM rate by setting a simple spike-rate threshold. 
%
To test this hypothesis, we designed a behavioral task to evaluate the necessity of auditory cortex on discrimination of both sound frequency and AM rate. 
% MAYBE: Caveats: Cortex might be more modulated by behavioral parameters, context. 

%DONE: Not totally sure about the wording here. 
\section{What types of stimulus features require cortical processing for efficient discrimination?}

We designed a behavioral task that allows measurement of both sound frequency discrimination performance and AM rate discrimination performance within the same behavioral session.
%
This behavioral paradigm, while challenging for animals to learn, allows the experimenter to compare the effect of one manipulation of cortical activity on both discrimination tasks. 
%
In general, this paradigm is designed to allow experimental comparison of the role of a brain region in two discrimination tasks. 
%
This permits comparison of the relative change in task performance between the two discriminations.
%
This is important because, even if a task could be performed by subcortical pathways in the absence of cortical input, it is unlikely that cortical inactivation will have no effect on performance of a specific task.
%
Therefore, it is important to compare the \emph{relative} effect of cortical inactivation on discrimination of more than one stimulus dimension. 

We focused heavily on the potential role of auditory cortex in sensory processing and stimulus feature extraction, but this is not the only possible way in which cortical circuits enable sensory-driven behaviors, especially in tasks that require flexibility. 
%
Lastly, in \ch{\Rev} we expanded our investigation into the relative role of cortical and subcortical pathways by reviewing the available literature to ask the questions: What kinds : what aspects of flexible, sensory-driven behaviors require sensory cortex? 
%
What aspects of these behaviors could potentially be performed by subcortical pathways alone?

\section{What aspects of stimulus-driven behavior require sensory cortex?}

We first attempted to define a framework for classifying tasks in which flexibility is required. 
%
We then review inactivation studies which suggest that cortex is not always required for flexible adaptations to changes in task demands. 
%
However, sensory cortex does appear to make some types of flexible behaviors, such as reversal of stimulus-action associations, faster. 
%
Also, there is evidence to support the idea that sensory cortex is required to perform discriminations on more complex features of stimuli. 
%
Some flexible behaviors don't appear to be possible after cortical lesions, such as those involving selective attention, suggesting a critical role of cortical circuits in these types of tasks. 


