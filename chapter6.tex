\chapter{Conclusion}

%% Main goals
%This dissertation revolves around the following questions: 
% The work in this dissertation attempts to address the following questions: 
% \begin{itemize}
% \item What are the pathways and brain regions necessary to perform some types of sound-driven behavior? 
% \item How is auditory information represented along these pathways, and what transformations are done on this information?
% \item What is the purpose of this transformation? Are transformed representations of stimuli necessary for task performance? 
% \item What is the role of sensory cortex in flexible behavior more generally? For what types of behaviors are the computations performed by cortical circuits really required?
% \end{itemize}

The work presented in this dissertation attempts to address the following questions: 
%
What are the neural pathways and circuits which enable animals to perform actions in response to sounds in order to get reward? 
%
How is auditory information represented along these pathways, and what transformations are done on this information?
%
Are transformed representations of stimuli necessary for task performance?
%
For what types of behaviors are the computations performed by cortical circuits really required?
%
While much remains to be done to address these questions, the work presented here makes progress in several key ways. 
%

First, the experiments in \ch{\Thstr} provide a framework for comparing the representation of stimulus features in two parallel pathways that converge on a common target region. 
%
In this case, we compared the representation of sound frequency and temporal amplitude modulation in the parallel thalamostriatal and corticostriatal pathways, which converge on a region of the striatum shown in \ch{\Musc} to play a key role in a behavior requiring discrimination of sound frequency. 
%
Second, the behavioral task presented in \ch{\Amod} provides a potential method for teasing apart the relative contribution of cortical and subcortical pathways to discrimination of different stimulus features. 
%
Third, \ch{\Rev} provides a starting point for investigation of the additional ways that sensory cortex could contribute to flexible behavior beyond simply processing complex stimulus features not explicitly represented at the periphery.


\section{Pathways and regions required for sound-driven behavior}
We began by investigating the role of the posterior striatum in sound frequency discrimination. 
%
We chose to investigate this region for three reasons. 
%
First, a study published prior to the start of this dissertation work established that animals were able to perform a previously-learned sound frequency-categorization task even after near-complete lesion of auditory cortex, but not after lesion of auditory thalamus \citep{Gimenez2015}.
%
This suggested that a pathway involving one of the subcortical outputs of the auditory thalamus was involved in this task.
%
Additionally, a pair of studies suggested that the corticostriatal pathway, under normal conditions, is involved in a similar sound-frequency discrimination task \citep{Znamenskiy2013}, and that the synaptic strength between auditory cortex neurons and striatal neurons changes during learning in a way that could support the learned sound-action association \citep{Xiong2015}.
%
This evidence led us to hypothesize that the preservation of the ability to perform frequency discrimination after lesion of auditory cortex was because the parallel thalamostriatal pathway continued to provide frequency information to the posterior striatum.
%%%%%%%%%%

%%%%%%%%%%
We set out to address this hypothesis by first determining whether frequency discrimination is possible after inactivation of posterior striatum. 
%
In \ch{\Musc} we found that inactivation of posterior striatum severely impaired frequency discrimination performance, pointing to a key role for this area in the performance of this behavior. 
%
%We then attempted to address the role of the parallel thalamostriatal and corticostriatal pathways.
Neural projections from both auditory cortex and auditory thalamus converge on this region of the striatum, but it was unclear how these two pathways represented sound features.

\section{Representation of sound features in auditory inputs to the posterior striatum}
In \ch{\Thstr} we performed pathway-specific electrophysiology to investigate the representation of sound features in the corticostriatal and thalamostriatal pathways. 
%
We found that the two pathways encoded sound frequency with similar fidelity, but that the representation of temporal modulations in amplitude differed between the pathways. 
%
We hypothesized that the transformation from a temporal code for amplitude modulation to a rate-coded representation, where cortical neurons were much more likely to fire more spikes in response to a preferred rate, would better allow a downstream area to discriminate AM rate by setting a simple spike-rate threshold. 
%
To test this hypothesis, we designed a behavioral task to evaluate the necessity of auditory cortex for discrimination of both sound frequency and AM rate. 
% MAYBE: Caveats: Cortex might be more modulated by behavioral parameters, context. 

%DONE: Not totally sure about the wording here. 
\section{What types of stimulus features require cortical processing for efficient discrimination?}

The behavioral task presented in \ch{\Amod} allows measurement of both sound frequency discrimination performance and AM rate discrimination performance within the same behavioral session.
%
This behavioral paradigm, while challenging for animals to learn, allows the experimenter to compare the effect of one manipulation of cortical activity on both discrimination tasks. 
%
In general, this paradigm is designed to allow experimental comparison of the role of a brain region in two discrimination tasks. 
%
This permits comparison of the relative change in task performance between the two discriminations.
%
This is important because, even if a task could be performed by subcortical pathways in the absence of cortical input, it is unlikely that cortical inactivation will have no effect on performance of a specific task.
%
%Therefore, it is important to compare the \emph{relative} effect of cortical inactivation on discrimination of more than one stimulus dimension. 

We focused heavily on the potential role of auditory cortex in sensory processing and stimulus feature extraction, but this is not the only possible way in which cortical circuits enable sensory-driven behaviors, especially in tasks that require flexibility. 
%
We expanded our investigation into the relative role of cortical and subcortical pathways in \ch{\Rev} by reviewing the available literature to ask the questions: What aspects of flexible, sensory-driven behaviors require sensory cortex? 
%
What aspects of these behaviors could potentially be performed by subcortical pathways alone?

\section{What aspects of stimulus-driven behavior require sensory cortex?}

%The bulk of the work in this dissertation focuses on the potential role of sensory cortex in processing stimuli not explicitly represented at the periphery. 
%
In \ch{\Rev} we present evidence from the literature supporting the idea that stimulus representations are transformed along sensory pathways from the periphery to the cortex, and also that discrimination of stimulus features not explicitly represented at the periphery (such as amplitude- or frequency-modulation) suffers after lesion of sensory cortex.
%
However, in this chapter we also expand to consider the contributions of sensory cortex to behavioral flexibility, beyond its role simply in processing complex stimulus features. 
%
% However, sensory cortex does appear to make some types of flexible behaviors, such as reversal of stimulus-action associations, faster. 
For instance, reversal of stimulus-action associations appears to be somewhat impaired after lesion of sensory cortex.
%
Some flexible behaviors don't appear to be possible after cortical lesions, such as those involving selective attention, suggesting a critical role of cortical circuits in these types of tasks. 
%
This review provides a starting point for investigating the roles of cortical and subcortical pathways in more complex behavioral paradigms requiring shifting associations between stimuli, actions, and outcomes - paradigms which better approximate the types of natural behaviors required to survive in a changing environment. 
%The discrimination tasks presented in \textbf{Chapter II} and \textbf{IV} permit dissection of the role of neural pathways and regions in representing and performing discrimination of sound features. 


