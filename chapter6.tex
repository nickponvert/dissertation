\chapter{Conclusion}

The experiments in Chapter 2 indicated a crucial role for the posterior striatum during performance of an auditory discrimination task. 
%
Previous literature had suggested a role for the pathway between auditory cortex and the posterior striatum in the context of a similar auditory task, which is consistent with the results of the inactivation experiments.
%
However, a model in which the corticostriatal pathway is solely responsible for facilitating behaviors involving discriminations of sound frequency are not consistent with lesion studies, which show that animals have a remarkable ability to perform auditory decision-making tasks, even requiring rapid flexibility, without auditory cortex. 
%
We therefore investigated the relationship between the auditory corticostriatal pathway and the parallel thalamostriatal pathway, interested in how these two sources of auditory input to the striatum compare and how they contribute to auditory decision-making tasks under different conditions. 

We found that the thalamostriatal and corticostriatal pathways convey information about sound frequency to the striatum with similar fidelity, potentially explaining why animals were found to be able to perform frequency discrimination tasks without auditory cortex. 
%
We also found that the representation of other sound features differed. 
%
Specifically, we found that the corticostriatal and thalamostriatal pathways differed in their representation of AM rate, with corticostriatal neurons more likely to be tuned to particular AM rates, and less likely than thalamostriatal neurons to synchronize their firing to the envelope of the modulation.
%
% TODO: Cite Deutscher, whatever else.
Based on these observations, and studies which suggested that AM discrimination performance suffers after auditory cortical lesion, we hypothesized that this cortical representation of AM rate better allows downstream circuits to perform discrimination of AM rate.

To test this hypothesis, we designed a task to compare the effect of a cortical manipulation on both AM rate discrimination and frequency discrimination in the same behavior session. 
%
Although it was difficult for animals to learn this task, with sufficient training they were able to achieve high levels of performance on both discrimination tasks. 
%
Pilot inactivation studies in animals performing this task suggested that AC inactivation impairs both frequency and AM rate discrimination behavior, and allowed us to identify key areas for future improvement of this method that would better enable comparisons of the magnitude of the effect on the performance of each discrimination. 
%
Principally, future experiments should better leverage the flexibility of the two-alternative forced-choice paradigm by changing the stimulus sets in order to make each discrimination easier or harder, allowing better matching of pre-inactivation performance between the two modalities.
%



%I found that inactivation of posterior striatum causes task deficits, and that these task deficits can't easily be explained by motor deficits alone.
%
%This is consistent with the other experiments in the paper, which showed that activation of posterior striatal neurons does not directly cause head rotation but does cause behavioral biases if done while the animal is performing a task. 
%
%The results of this inactivation are also largely consistent with the results of \citet{Znamenskiy2015}, in that they point to a key role for the auditory striatum in the performance of frequency-discrimination tasks. 


%Striatal activation, in our hands, resulted in strong contralateral biases regardless of the preferred frequency of the striatal neurons. 

% TODO: Check supplemental of ref 10 in Lan's paper (I think Petr). 
Zador lab stimulated AC without frequency-specific and found a contralateral bias.

Both of these pieces of evidence point towards the MSNs in the pStr acting as drivers of contralateral choice. 

% Do we see these ``contralateral'' biases only because we are using a left/right task? What about a go/no-go task? 

How can behavioral and neurophysiological comparisons help reveal the role of a brain area?

If an area is directly making the decision, then the responses of neurons there should be different when animals make a different choice but the presented sounds are the same. 

Can test whether the pre-withdrawal activity encodes an animal's choice.

Both of these have the choice being variable because the stimulus is close to the categorization boundary. So you can also do reversal tasks where the stimulus-action association is clear but changes between contingencies. 

%%%% Stuff related to Thstr paper

% What evidence exists for synaptic plasticity at thalamostriatal synapses?
% Look at thalamoamygdala
It is not known whether connections between sensory thalamus neurons and neurons in the striatum are capable of the same type of learning-related plasticity observed in corticostriatal synapses \citep{Xiong2015}, although plasticity has been observed between intralaminar thalamus and striatum \citep{Parker2016}.
%
One study has shown that simultaneous stimulation of auditory corticostriatal and thalamostriatal neurons leads to long term potentiation at corticostriatal synapses but not at thalamostriatal synapses, suggesting that, if the thalamostriatal pathway is capable of synaptic plasticity, different rules may govern the changes. 
% TODO: I am here
Plasticity does occur between neurons in the thalamus and those in the amygdala following fear conditioning,
cocaine conditioning \citep{Rich2019}

% The recording experiments happened in untrained animals. What might happen if animals are trained before? Where would the changes potentially occur? 

% Do thalamostriatal and corticostriatal inputs converge on the same MSNs?

% What does it mean for the inputs onto MSNs if the MSNs in posterior striatum may not be the ones actually performing the selection? 

% TODO: Check out Brice B's paper showing that for simple stimuli the thalamic pathway takes over, but for complex stimuli the reaction times are longer and the cortical pathway seems required. 


% Might cortex just be serving to reduce complex stimuli to simple rate-tuned representations? 

%%%%%%%%%%%%% Structure %%%%%%%%%%%%%%%%%
% Cortico-striatal involved, but on a similar task cortex is not necessary once animal has learned. We showed that striatum is required for task performance.
% We then turned to the thalamostriatal pathway, and compared the sound information being sent via that pathway and the corticostriatal pathway. Freq same, AM different.
% This raises the question of whether the thalamostriatal pathway is sufficient for the performance of some kinds of tasks, but not sufficient for others. A review of the literature related to the role of cortex in behavioral flexibility supports this idea.
% We designed and implemented a behavioral task that would allow the effect of a manipulation (such as chemical inactivation of AC) on performance of two discrimination tasks, frequency and AM rate.

% Overall in this dissertation we found:
%% Striatum is for sure a part of the system for performing sound frequency discriminaton discrimination behavior
%
%% The thalamostriatal and corticostriatal pathways send complementary information about sounds, but it isn't super clear why.
%
%% A review of the role of cortex in flexible behavior suggests that there are some classes of behaviors that can be performed without cortex, but there are some classes of beahviors where cortex is necessary.
%
%% We developed a task that can be used to evaluate the side-by-side effects on discrimination of different sound features.
%

