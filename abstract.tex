\abstract{
% Making decisions about sound is important for many species
For many species, the ability to use information about sound to guide behavior is important for obtaining rewards such as food, mates, or respite from danger.
% The neural pathways in the brain that associate sounds with actions are not well understood
The neural pathways that associate sounds with actions that will produce reward are not well understood. 
% The auditory pathways through the striatum represent a potential circuit for associating sounds with actions
The striatum, a brain region which receives sound information from the auditory cortex and auditory thalamus, and which can drive behavior via its outputs in the basal ganglia, represents a potential key hub in the pathway between sound and action. 

In this thesis, I investigate the role of striatal circuits in the context of sound-guided behavior, and compare parallel corticostriatal and thalamostriatal pathways that provide auditory information to the striatum.
% 2016astr stuff
In \ch{\Musc}, I inactivate striatum while animals perform a frequency discrimination task, and find that this manipulation leads to a strong deficit in task performance. 
% 2018thstr stuff
In \ch{\Thstr}, I compare the information about sounds sent to the striatum from the auditory cortex and auditory thalamus.
I find that, while these pathways convey frequency information to the striatum with similar fidelity, they have different representations of temporal modulations in sound amplitude. 
% Amod stuff
In \ch{\Amod} I then present a behavioral task that could be used to evaluate the relative contribution of the cortico-striatal and thalamo-striatal pathways to discriminations of different sound features.
%Review stuff
Finally, \ch{\Rev} provides a review of the role of cortical and subcortical pathways in other behavioral contexts, especially in tasks that require rapid flexibility. 

%Need to end with this statement if you're working from papers published or soon to be published.
This dissertation includes previously published and unpublished co-authored material.
}
